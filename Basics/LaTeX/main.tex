%%%%%%%%%%%%%%%%%%%%%%%%%%%%%%%%%%%%%%%%%%%%%%%%%%%%%%%%%%%%%%%
%
% Welcome to Overleaf --- just edit your LaTeX on the left,
% and we'll compile it for you on the right. If you open the
% 'Share' menu, you can invite other users to edit at the same
% time. See www.overleaf.com/learn for more info. Enjoy!
%
%%%%%%%%%%%%%%%%%%%%%%%%%%%%%%%%%%%%%%%%%%%%%%%%%%%%%%%%%%%%%%%
% basic setup
\documentclass[12pt, letterpaper]{article}

% graphics package
\usepackage{graphicx} %LaTeX package to import graphics
\graphicspath{{images/}} %configuring the graphicx package
% asmsmath package
\usepackage{amsmath}% For the equation* environment

% info
\title{My first LaTeX document}
\author{Hubert Farnsworth\thanks{Funded by the Overleaf team.}}
\date{August 2022}
\begin{document}

\tableofcontents
% abstract
\begin{abstract}
    I love abstracts because abstracts are abstracts
\end{abstract}

First document. This is a simple example, with no 
extra parameters or packages included.

% bold, underline and etalic
\section{Basic styling}
Some of the \textbf{greatest}
discoveries in \underline{science} 
were made by \textbf{\textit{accident}}.

Some of the greatest \emph{discoveries} in science 
were made by accident.
\textit{Some of the greatest \emph{discoveries} 
in science were made by accident.}
\textbf{Some of the greatest \emph{discoveries} 
in science were made by accident.}

\section{Figures}
% figures
\begin{figure}[h]
    \centering
    \includegraphics[width=0.75\textwidth]{universe}
    \caption{A nice plot.}
    \label{fig:universe}
\end{figure}

As you can see in figure \ref{fig:universe}, the function grows near the origin. This example is on page \pageref{fig:universe}.

\section{Lists}
% lists
\begin{itemize}
  \item The individual entries are indicated with a black dot, a so-called bullet.
  \item The text in the entries may be of any length.
\end{itemize}

\begin{enumerate}
  \item This is the first entry in our list.
  \item The list numbers increase with each entry we add.
\end{enumerate}

\section{math}
% math
\subsection{inline math}
% intline math
In physics, the mass-energy equivalence is stated 
by the equation $E=mc^2$, discovered in 1905 by Albert Einstein.

\subsection{display math}
% display math
The mass-energy equivalence is described by the famous equation \[ E=mc^2\] discovered in 1905 by Albert Einstein.

\subsection{More examples}
% more math examples
Subscripts in math mode are written as $a_b$ and superscripts are written as $a^b$. These can be combined and nested to write expressions such as

\[ T^{i_1 i_2 \dots i_p}_{j_1 j_2 \dots j_q} = T(x^{i_1},\dots,x^{i_p},e_{j_1},\dots,e_{j_q}) \]
 
We write integrals using $\int$ and fractions using $\frac{a}{b}$. Limits are placed on integrals using superscripts and subscripts:

\[ \int_0^1 \frac{dx}{e^x} =  \frac{e-1}{e} \]

Lower case Greek letters are written as $\omega$ $\delta$ etc. while upper case Greek letters are written as $\Omega$ $\Delta$.

Mathematical operators are prefixed with a backslash as $\sin(\beta)$, $\cos(\alpha)$, $\log(x)$ etc.

\subsection{amspath package}
\subsubsection{First example}

The well-known Pythagorean theorem \(x^2 + y^2 = z^2\) was proved to be invalid for other exponents, meaning the next equation has no integer solutions for \(n>2\):

\[ x^n + y^n = z^n \]

\subsubsection{Second example}

This is a simple math expression \(\sqrt{x^2+1}\) inside text. 
And this is also the same: 
\begin{math}
\sqrt{x^2+1}
\end{math}
but by using another command.

This is a simple math expression without numbering
\[\sqrt{x^2+1}\] 
separated from text.

This is also the same:
\begin{displaymath}
\sqrt{x^2+1}
\end{displaymath}

\ldots and this:
\begin{equation*}
\sqrt{x^2+1}
\end{equation*}

\section{Structure}
\subsection{Basics}

After our abstract we can begin the first paragraph, then press ``enter'' twice to start the second one.

This line will start a second paragraph.

I will start the third paragraph and then add \\ a manual line break which causes this text to start on a new line but remains part of the same paragraph. Alternatively, I can use the \verb|\newline|\newline command to start a new line, which is also part of the same paragraph.

\subsection{tables}

\begin{center}
\begin{tabular}{|c|c|c|} 
 \hline
 cell1 & cell2 & cell3 \\ 
 cell4 & cell5 & cell6 \\ 
 cell7 & cell8 & cell9 \\ 
 \hline
\end{tabular}
\end{center}

\subsubsection{Detailed table}
Table \ref{table:data} shows how to add a table caption and reference a table.
\begin{table}[h!]
\centering
\begin{tabular}{||c c c c||} 
 \hline
 Col1 & Col2 & Col2 & Col3 \\ [0.5ex] 
 \hline\hline
 1 & 6 & 87837 & 787 \\ 
 2 & 7 & 78 & 5415 \\
 3 & 545 & 778 & 7507 \\
 4 & 545 & 18744 & 7560 \\
 5 & 88 & 788 & 6344 \\ [1ex] 
 \hline
\end{tabular}
\caption{Table to test captions and labels.}
\label{table:data}
\end{table}

\end{document}
